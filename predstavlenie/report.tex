\documentclass[14pt, a4paper]{extarticle}
\usepackage{polyglossia} % Поддержка многоязычности (fontspec подгружается автоматически)
\setmainlanguage[babelshorthands=true]{russian} % Язык по-умолчанию русский с поддержкой приятных команд пакета babel
\setotherlanguage{english} % Дополнительный язык = английский (в американской вариации по-умолчанию)
\setmonofont{CMU Typewriter Text} % моноширинный шрифт
\newfontfamily\cyrillicfonttt{CMU Typewriter Text} % моноширинный шрифт для кириллицы
\defaultfontfeatures{Ligatures=TeX} % стандартные лигатуры TeX, замены нескольких дефисов на тире и т. п. Настройки моноширинного шрифта должны идти до этой строки, чтобы при врезках кода программ в коде не применялись лигатуры и замены дефисов
\setmainfont{PT Astra Serif} % Шрифт с засечками
\newfontfamily\cyrillicfont{PT Astra Serif}[Script = Cyrillic] % Шрифт с засечками для кириллицы
\setsansfont{CMU Sans Serif} % Шрифт без засечек
\newfontfamily\cyrillicfontsf{CMU Sans Serif} % Шрифт без засечек для кириллицы

\usepackage{amsmath}
\usepackage{unicode-math}
\usepackage{float}
\usepackage[figurename=Рисунок]{caption, subcaption}
\usepackage{geometry}
\usepackage{graphicx}
\usepackage{titlesec}
\usepackage{enumitem}
\usepackage{indentfirst}
\usepackage[colorlinks=true,linkcolor=black,urlcolor=black,citecolor=black]{hyperref}
\usepackage{listings}
\usepackage{xcolor}
\usepackage{longtable}
\usepackage[lighttt]{lmodern}
\usepackage{appendix}
\graphicspath{{./images/}}
\geometry{left=30mm,
          right=10mm,
          top=20mm,
          bottom=20mm}

\usepackage{microtype}

\usepackage{array}
\usepackage{multirow}

\pretolerance=5000
\tolerance=9000
\hyphenpenalty=1000
\emergencystretch=0pt

\usepackage{fontspec}
\setmonofont[
  Contextuals={Alternate},
  Scale=MatchLowercase
]{Fira Mono for Powerline}
\lstdefinestyle{bash}{
    basicstyle=\ttfamily,
    framerule=10pt,
    breakatwhitespace=false,
    breaklines=true,
    captionpos=b,
    keepspaces=true,
    showspaces=false,
    showstringspaces=false,
    showtabs=false,
    tabsize=2,
    columns=fullflexible,
    literate={-}{-}1
}

\lstset{style=bash}
\setmathfont{Latin Modern Math}

\setlist[itemize]{noitemsep}
\setlist[enumerate]{noitemsep}

\renewcommand{\figurename}{Рисунок}
\renewcommand{\lstlistingname}{Исходный код}

\DeclareSymbolFont{cyrletters}{\encodingdefault}{\familydefault}{m}{it}
\newcommand{\makecyrmathletter}[1]{%
\begingroup\lccode`a=#1\lowercase{\endgroup
\Umathcode`a}="0 \csname symcyrletters\endcsname\space #1
}
\count255="409
\loop\ifnum\count255<"44F
\advance\count255 by 1
\makecyrmathletter{\count255}
\repeat

\definecolor{mygray}{rgb}{0.3,0.3,0.3}

\let\oldtexttt\texttt
\renewcommand{\texttt}[1]{\textcolor{mygray}{\oldtexttt{#1}}}

\renewcommand{\thefigure}{\arabic{figure}}
\renewcommand{\theequation}{\arabic{figure}}
\renewcommand{\labelenumii}{\theenumii}
\renewcommand{\theenumii}{\theenumi.\arabic{enumii}.}

\date{\today}

\begin{document}

\righthyphenmin=4
\lefthyphenmin=3

\begin{titlepage}
    \begin{center}
        Федеральное государственное автономное образовательное учреждение высшего
        образования\par
        НАЦИОНАЛЬНЫЙ ИССЛЕДОВАТЕЛЬСКИЙ УНИВЕРСИТЕТ\par ВЫСШАЯ ШКОЛА ЭКОНОМИКИ\par
        \vspace{1cm}
        Московский институт электроники и математики им. А.Н. Тихонова \\
        \vspace{1cm}
        Проект 19106 \\
        \vspace{2cm}
    \end{center}

    \begin{center}
        ТЕХНИЧЕСКОЕ ЗАДАНИЕ \\
        на разработку информационной системы \\
        \textbf{Mekstack: Приватное Облако} \\
    \end{center}
    \vfill

    \begin{minipage}[t]{0.5\textwidth}
        \begin{center}
        УТВЕРЖДАЮ
        \end{center}
        \begin{flushleft}
        Рыбаков Петр Владимирович\\
        Руководитель направления\\

        \vspace{0.5cm}
        Подпись
        \vspace{0.5cm}

        \today
        \end{flushleft}
    \end{minipage}%

    \vspace{2cm}

    \begin{minipage}[t]{0.5\textwidth}
        \begin{center}
        СОГЛАСОВАНО
        \end{center}
        \begin{flushleft}
        Емельяненко Максим Владимирович\\
        Руководитель проекта

        \vspace{0.5cm}
        Подпись
        \vspace{0.5cm}

        \today
        \end{flushleft}
    \end{minipage}%

    \vspace{2cm}
    \center
    Москва, 2023
\end{titlepage}

\pagenumbering{roman}

\tableofcontents

\pagebreak

\pagenumbering{arabic}

\section{Актуальность проекта}

В настоящее время большинство проектов требуют наличия специализированной виртуальной инфраструктуры, предоставляемой учебным заведением. Вопреки очевидным преимуществам, которые дает применение виртуальных вычислительных машин, это вызывает возникновение нового ряда требований со стороны пользователей, удовлетворение которых является целью данного проекта.

В рамках образовательного процесса студентов МИЭМ был проведен анализ, в результате которого были выявлены следующие требования к инфраструктуре и функционалу информационной системы (ИС), предоставляющей виртуальную инфраструктуру:

\begin{enumerate}

\item Разработка и внедрение интуитивно понятной и оперативной системы документации с функцией поиска, способной максимально раскрыть возможности частного облачного сервиса и обеспечить поддержку пользователей в освоении разнообразных методик облачных решений, включая наглядные примеры применения.
\item Создание высокодоступной системы, обладающей функционалом автоматического восстановления после сбоев (в том числе после отключения электроэнергии на территории всего кампуса) для обеспечения непрерывности работы облачных сервисов.
\item Интеграция современных технологических решений, используемых в передовых IT-компаниях, для предоставления студентам актуального практического опыта.
\item Реализация механизмов, позволяющих пользователям самостоятельно решать возникающие проблемы с минимальной зависимостью от службы технической поддержки и снижающих нагрузку на персонал, ответственный за обработку запросов.
\item Обеспечение комплексного мониторинга состояния виртуальных машин и облачных сервисов с возможностью настройки пользовательских метрик и системы оповещений для своевременного реагирования на инциденты.
\item Разработка интерфейса или системы управления, которая дает возможность пользователям самостоятельно управлять ресурсами в облаке без вмешательства сторонних специалистов.
\item Внедрение возможности подключения к виртуальной инфраструктуре с использованием доменной авторизации Высшей школы экономики.
\item Предоставление функционала для самостоятельного создания клиентов OpenID с целью настройки доменной авторизации в собственных информационных системах.
\item Реализация возможности публикации разработанных информационных систем в сети Интернет.
\item Обеспечение доступа к оперативной технической поддержке для решения возникающих вопросов и отладки проблем.

\end{enumerate}

\subsubsection*{Преподавателям МИЭМ нужно:}

\begin{enumerate}
\item Обеспечить функцию автоматизации стандартных операций для проведения лабораторных работ, включая автоматизированное создание виртуальных машин, но не ограничиваясь им.
\end{enumerate}

\subsubsection*{Сотрудникам МИЭМ требуется:}

\begin{enumerate}
\item Реализация функции контроля использования и/или недостачного использования запрошенных ресурсов информационных технологий.
\item Создание системы регистрации событий, которая позволит фиксировать все действия и изменения в виртуальной среде, что станет основой для проведения аудита и будет способствовать эффективной диагностике проблем.
\item Внедрение системы оперативного реагирования на инциденты информационной безопасности.
\item Организация студенческой инфраструктуры, функционирующей в изоляции от критически важных элементов инфраструктуры Высшей школы экономики.
\item Способ снижения нагрузки на персонал, отвечающий за обслуживание виртуальных машин.
\end{enumerate}

\section{Назначение результатов проекта}

\begin{enumerate}
\item Автоматизация процесса обработки заявок для сокращения времени их рассмотрения и исключения необходимости ручного ввода данных.
\item Внедрение системы самообслуживания для пользователей, позволяющей им самостоятельно управлять выделенными ресурсами, что сократит нагрузку на персонал техподдержки.
\item Создание комплексной системы мониторинга, обеспечивающей постоянный контроль за состоянием виртуальных машин, облачной инфраструктуры и информационных систем как сервисных, так и пользовательских.
\item Реализация функции логирования всех значимых событий для обеспечения возможности аудита и упрощения диагностики проблем.
\item Упрощение эксплуатации облачной инфраструктуры за счет разработки инструкций по работе с виртуальными машинами и облачными сервисами.
\item Улучшение пользовательского интерфейса с целью сделать его более интуитивно понятным и удобным для пользователей.
\item Интеграция с системами Московского института электроники и математики и личным кабинетом Высшей школы экономики для создания единого информационного пространства.
\item Оптимизация использования ресурсов путем идентификации и обработки заброшенных виртуальных машин и дисков.
\end{enumerate}

\section{Цель разработки}

\begin{enumerate}
\item Создание автоматизированного, легко конфигурируемого и расширяемого кластера серверов на основе проекта OpenStack Kayobe.
\item Создание системы централизованного управления и мониторинга ресурсов.
\item Интеграция с личным кабинетом ВШЭ.
\item Подготовка подробной документации (документация администратора, архитектурная документация в формате Docs as a Code) и пользовательских мануалов, настройка Elasticsearch для поиска по документации.
\item Создание VPN сервиса для подключения студентов к сети облака.
\item Создание ИС для выделения доменов и создания tls-сертификатов пользователям.
\item Интеграция с Ceph SDS для получения высокоэффективного, устойчивого к отказам и самовосстанавливающегося хранилища данных.
\item Адаптация и внедрение OpenStack Swift, системы хранения, оптимизированной для работы с неструктурированными данными.
\item Адаптация и внедрение NFSaaS (Manilla), чтобы пользователи облака могли создавать и управлять файловыми шарами (например, NFS или CIFS) без необходимости управлять низлежащими серверами файлового хранения или NAS (network-attached storage) устройствами.
\item Адаптация и внедрение DBaas (Trove), чтобы пользователи могли развертывать и управлять базами данных без необходимости заботиться о поддержке и управлении физическими серверами.
\item Разработка прокси-сервиса для кэширования и предоставления общего доступа к артефактам, который обеспечит оптимизацию доступа к зависимостям и библиотекам для ускорения процессов сборки и развертывания.
\item Разработать бота на базе ChatGPT для умной коммуникации с пользователем.
\item Реализовать сервисную сетку (service mesh) на базе Istio внутри кластера Kubernetes для обеспечения управления трафиком, безопасности, и наблюдаемости ИС с целью повышения надёжности и управляемости микросервисной архитектуры.
\item Предоставить централизованный, защищённый сервис для хранения, управления и контроля доступа к секретам и конфиденциальной информации, используя HashiCorp Vault как сервис в облачной инфраструктуре.
\item Реализовать облачный сервис удостоверяющего центра (IdP) для централизованного управления идентификацией и аутентификацией пользователей с использованием интеграции с системой Vault.
\item Автоматизировать процесс создания и обновления образов операционных систем для использования в облачной инфраструктуре, обеспечивая постоянную актуализацию безопасности и оптимизацию ресурсов.
\item Разработать и реализовать комплексные сценарии восстановления после сбоев (disaster recovery), гарантирующие минимальное время простоя и минимальную потерю данных в случае катастрофических событий или сбоев инфраструктуры.
\item Внедрить системы обнаружения и предотвращения вторжений для мониторинга и защиты облачной инфраструктуры от неавторизованного доступа, атак и уязвимостей.
\item Развернуть систему SIEM для анализа событий безопасности в реальном времени и управления инцидентами, повышая общую безопасность и соответствие нормативным требованиям.
\item Предоставить управляемый сервис Kubernetes, позволяющий пользователям легко развертывать, управлять и масштабировать контейнеризованные приложения с полной поддержкой и автоматизацией задач инфраструктуры.
\item Разработать и интегрировать модуль удостоверяющего центра для Keystone, который позволит упростить и централизовать доменную аутентификацию и авторизацию пользователей в Mekstack.
\end{enumerate}

\section{Требования к результатам проекта}

\subsection{Интеграция авторизации с личным кабинетом}

Нужно разработать и интегрировать модуль удостоверяющего центра для Keystone, который позволит упростить и централизовать аутентификацию и авторизацию пользователей в Mekstack.

\subsubsection*{Триггер начала работ:}
Успешная имплементация Apache Kafka в личном кабинете МИЭМ.

\subsubsection*{Требования к сервису:}

Сервис должен создавать и изменять пользователей и проекты в Keystone аналогично их сущностям в кабинете ВШЭ.

\subsubsection*{Задачи:}

\begin{enumerate}
\item Разработать пользовательский модуль аутентификации Keystone, который будет использовать данные проектов пользователя из OpenID.
\item Написать патч для Keystone.
\item Написать пайплайн сборки и публикации образов.
\item Написать пайплайн для периодического слияния нашей версии Keystone с основной кодовой базой Keystone.
\item Написать юнит-тесты для патча.
\item Настроить логирование для нового модуля аутентификации.
\item Реализовать канареечное развертывание для поэтапного внедрения модуля.
\item Разработать пайплайн для сборки образов Keystone с новым модулем.
\item Автоматизировать публикацию образов в реестр.
\item Обеспечить версионирование для возможности отката и ведения аудита.
\item Интегрировать пайплайн с существующими процессами CI/CD.
\item Провести аудит безопасности нового модуля аутентификации.
\item Устранить уязвимости, обнаруженные в ходе аудита.
\item Гарантировать соответствие соответствующим стандартам безопасности и лучшим практикам.
\item Разработать стратегию внедрения нового модуля в существующие среды Keystone.
\item Создать инструмент миграции для пользователей, использующих предыдущий метод аутентификации.
\item Протестировать процесс миграции в контролируемых условиях.
\end{enumerate}

\subsection{Документация облака для разработчиков}

Наиболее возможно полная документация функциональности облака с узконаправленной информацией для разработчиков.

\subsubsection*{Для этого должна быть следующая документация:}

\begin{enumerate}
\item Документация архитектуры всех сервисов
\item Документация архитектуры облака
\item Схема сетевых потоков (L3)
\item Схема сетевой связанности (L2)
\item Схема расположения оборудования в стойках (L1)
\end{enumerate}

\subsection{Сервис публикации пользовательских HTTP серверов \\(NATaaS)}

Создание ИС на базе Envoy для выделения доменов и создания TLS сертификатов пользователям.
Реализация данного требования поможет существенно снизить временные затраты по сравнению с выполнением аналогичных действий вручную.

\subsubsection*{Ключевые функции NATaaS}

\begin{enumerate}
\item \textbf{Перенаправление запросов (Request Forwarding):} Основная функция обратного прокси, обработка входящих HTTP запросов и направление их к соответствующим бэкенд-серверам.
\item \textbf{Балансировка нагрузки (Load Balancing):} Распределение входящего сетевого трафика между группой бэкенд-серверов для предотвращения перегрузки отдельных серверов.
\item \textbf{TLS Termination:} Расшифровка входящих TLS соединений на прокси для снижения вычислительной нагрузки на бэкенд-серверы.
\item \textbf{Мультиплексирование соединений:} Снижение нагрузки на бэкенд-серверы за счет повторного использования соединений для нескольких клиентских запросов.
\item \textbf{HTMX SPA и gRPC HTTP/2:} Должна быть обеспечена поддержка HTMX SPA и gRPC HTTP2 для взаимодействия с бекендами и сервисом обнаружения сервисов
\end{enumerate}

\subsubsection*{Функции для оптимизация производительности}

\begin{enumerate}
\item \textbf{Кэширование (Caching):} Хранение копий часто запрашиваемых ресурсов для снижения нагрузки на сервер и улучшения времени отклика.
\item \textbf{Сжатие (Compression):} Сжатие исходящего контента для ускорения передачи данных клиенту.
\item \textbf{Поддержка HTTP/3 и gRPC:} Поддержка современных протоколов для эффективной коммуникации.
\end{enumerate}

\subsubsection*{Функции безопасности}

\begin{enumerate}
\item \textbf{Web Application Firewall (WAF):} Защита от общих уязвимостей веб-приложений и атак, таких как SQL-инъекция и межсайтовый скриптинг (XSS).
\item \textbf{Защита от DDoS-атак:} Снижение эффекта распределенных атак типа "отказ в обслуживании" с автоматизированной фильтрацией трафика.
\item \textbf{API Throttling и Rate Limiting:} Предотвращение злоупотреблений и управление трафиком путем ограничения числа запросов от пользователя за определенный временной промежуток.
\item \textbf{Фильтрация IP-адресов (IP Filtering):} Создание белых и черных списков IP-адресов или диапазонов для контроля доступа к бэкенд-сервисам.
\end{enumerate}

\subsubsection*{Интеграция и настройка}

\begin{enumerate}
\item \textbf{Поддержка пользовательских доменных имен:} Возможность для клиентов использовать свои собственные доменные имена для услуги обратного прокси.
\item \textbf{Маршрутизация на основе пути (Path-Based Routing):} Маршрутизация запросов на основе пути или имени хоста в запросе.
\item \textbf{Модификация заголовка хоста (Host Header Modifications):} Изменение заголовка хоста в HTTP-запросах при их прохождении через прокси.
\end{enumerate}

\subsubsection*{Мониторинг и управление}

\begin{enumerate}
\item \textbf{Логирование и мониторинг:} Подробные журналы доступа и реальный мониторинг производительности прокси.
\item \textbf{Оповещения и уведомления:} Автоматизированные уведомления о проблемах, таких как простои сервера или высокая задержка ответов.
\item \textbf{Аналитика и отчетность:} Предоставление данных о моделях трафика, поведении пользователей и производительности API.
\end{enumerate}

\subsubsection*{Инструменты для разработчиков}

\begin{enumerate}
\item \textbf{Интеграция с CI/CD:} Инструменты для интеграции с системами непрерывной интеграции и непрерывной доставки.
\end{enumerate}

\subsubsection*{Масштабируемость и надежность}

\begin{enumerate}
\item \textbf{Автоматическое масштабирование (Auto Scaling):} Автоматическая корректировка числа экземпляров обратного прокси-сервиса в зависимости от нагрузки.
\item \textbf{Высокая доступность (High Availability):} Разработка системы с учетом отказоустойчивости и обеспечение доступности ИС через резервирование и дублирование.
\item \textbf{Корректное переключение при отказе (Graceful Failover):} В случае сбоя сервера автоматическое перенаправление трафика на здоровые серверы без потери запросов.
\end{enumerate}

\subsubsection*{Интеграционные задачи}

\begin{enumerate}
\item \textbf{Интеграция с Fail2ban:} Должна быть обеспечена интеграция с Fail2ban для автоматического блокирования подозрительных IP адресов и предотвращения несанкционированного доступа.
\item \textbf{Интеграция с SIEM:} Должны быть настроены механизмы для сбора, анализа и сохранения логов в системе SIEM для обеспечения возможности мониторинга безопасности и управления инцидентами.
\item \textbf{Архитектура Service Discovery:} Должна быть реализована архитектура обнаружения ИС для автоматического обновления маршрутов и конфигураций в реальном времени, исключая Single Point of Failure.
\item \textbf{TLS авторегистрация:} Должна быть предусмотрена функция автоматической регистрации и обновления TLS сертификатов для пользовательских доменов.
\item \textbf{Добавление пользовательских доменов:} Должна быть возможность для пользователей добавлять собственные доменные имена в сервис.
\item \textbf{Лизинг поддоменов:} Должна быть предоставлена функциональность лизинга наших поддоменов пользователям с возможностью управления и настройки.
\item \textbf{Управление правилами форвардинга:} В веб-интерфейсе должна быть предусмотрена функция управления правилами форвардинга доменов пользователей.
\item \textbf{Интеграция с OpenStack:} Должна быть реализована интеграция с OpenStack для автоматического форвардинга трафика на инфраструктуру проектов.
\item \textbf{DDoS защита:} Должны быть реализованы механизмы защиты от DDoS-атак.
\item \textbf{OpenID авторизация:} Должна быть внедрена система доменной OpenID авторизации с использованием edu.hse.ru.
\item \textbf{Интеграция с Grafana:} Должна быть предусмотрена интеграция iframe с Grafana для отображения информации о статусе и нагрузке поддоменов пользователя.
\end{enumerate}


\subsection{Сервис авторизации и аутентификации пользователей\\ (Hashicorp Vault)}

Должен быть реализован облачный сервис удостоверяющего центра (IdP) для централизованного управления идентификацией и аутентификацией пользователей с использованием интеграции с системой Vault.
У пользователей должна быть возможность использовать Vault IdP в своих сервисах для осуществления доменной авторизации.

\subsubsection*{Для этого должны быть реализованы следующие функции:}

\begin{enumerate}
\item Синхронизация данных с кабинетом пользователя для актуализации информации о проектах пользователей.
\item Шифрование всех данных для обеспечения безопасности конфиденциальной информации.
\item Обеспечение высокой доступности ИС для непрерывной работы пользователей.
\item Использование алгоритма RAFT для репликации данных, гарантирующего их надежность и согласованность.
\item Предоставление пользователям доступа к API для интеграции с внешними приложениями и ИС.
\item Реализация системы единого входа (SSO) с использованием доменной OpenID авторизации по auth.hse.ru для упрощения процесса входа.
\item Возможность создания пользователями своих OpenID клиенты для авторизации в различных ИС.
\item Регулярное создание бекапов данных для обеспечения их сохранности.
\item Тестирование процедур восстановления из бекапов для проверки надежности соответствующих механизмов.
\item Внедрение практик непрерывной интеграции и непрерывного развертывания (CI/CD) для ускорения процессов разработки и обеспечения качества.
\item Использование Gitops для автоматизации и управления инфраструктурой через git.
\item Развертывание ИС в Kubernetes для улучшения управляемости и масштабируемости.
\item Обеспечение безопасности ИС и выполнение обязательств по обеспечению безопасности (Security Assurance).
\item Создание обучающих материалов и руководств для пользователей для повышения их осведомленности и умения пользоваться сервисом.
\item Автоматическое масштабирование ИС с использованием инструментов Kubernetes HPA (Horizontal Pod Autoscaler) для адаптации к изменяющейся нагрузке.
\item Внедрение сервис-мешей, таких как Istio, для управления трафиком, повышения отказоустойчивости и масштабируемости ИС.
\end{enumerate}

\subsection{Monitoring}

Помимо разработки новых ИС также нужно осуществлять мониторинг текущих.

\subsubsection*{Для этого нужно будет выполнить следующее задачи:}

\begin{enumerate}
\item Настройка и интеграция Grafana для визуализации данных мониторинга.
\item Использование Prometheus для сбора метрик с различных ИС.
\item Интеграция с OpenStack для мониторинга его ключевых компонентов.
\item Настройка системы Alerting для создания и управления алертами, включая прогнозирующие сбои.
\item Для SNEEDaaS мониторить следующие метрики:
    \begin{enumerate}
    \item “Здоровье” балансировщика нагрузки.
    \item Доступ из публичной сети.
    \item Доступ из внешней сети.
    \item Задержку в сети.
    \end{enumerate}
\item Для DNoS-HA мониторить следующие метрики:
    \begin{enumerate}
    \item “Здоровье” балансировщика нагрузки.
    \item Доступ из публичной сети.
    \item Задержку в сети.
    \item Состояние основных компонентов OpenStack.
    \item Срок действия сертификатов.
    \end{enumerate}
\item Для VPNaaS мониторить следующие метрики:
    \begin{enumerate}
    \item Производительность ИС.
    \item Количество свободных IP-адресов.
    \end{enumerate}
\item Для Vault мониторить следующие метрики:
    \begin{enumerate}
    \item Статус ИС.
    \item Количество пользователей.
    \item Частоту входа в систему.
    \item Синхронизацию заданий маппинга.
    \item Дату последнего запуска задания.
    \item Время выполнения задания.
    \end{enumerate}
\end{enumerate}

\subsection{Monitoring as a Service}

У пользователя должна быть возможность мониторинга состояния его виртуальной инфраструктуры.

\subsubsection*{Для этого нужно реализовать следующие функции:}

\begin{enumerate}
\item Мониторинг в реальном времени для непрерывного контроля за системами и приложениями.
\item Сбор показателей производительности для анализа и оптимизации работы систем.
\item Настраиваемые информационные панели, позволяющие пользователям создавать дашборды согласно их потребностям.
\item Настраиваемые оповещения для своевременного реагирования на критические события или проблемы.
\item Хранение исторических данных в состоянии для анализа тенденций и паттернов поведения системы.
\item Обнаружение аномалий для раннего выявления нестандартного поведения или потенциальных проблем.
\item Мониторинг безопасности для контроля за угрозами и уязвимостями системы.
\item Отчетность по политикам и соответствию для проверки соответствия виртуальных машин организационным политикам и создания отчетов для аудита соответствия.
\item Масштабируемая архитектура, для своевременной адаптации сервиса к изменяющимся требованиям без потери производительности.
\item Механизмы отказоустойчивости для поддержания работы ИС мониторинга в случае сбоев системы.
\item Интеграция с системами обнаружения вторжений для повышения уровня безопасности мониторинга.
\item Настраиваемые интервалы мониторинга, позволяющие пользователям определять частоту проверок для их виртуальных машин в соответствии с их потребностями.
\item Портал самообслуживания для добавления и управления виртуальными машинами пользователями без необходимости прямого вмешательства IT-персонала.
\item API для автоматизации задач мониторинга и интеграции с другими системами, такими как инструменты предоставления услуг или оркестрирования.
\end{enumerate}


\subsection{Кеширующее зеркало часто используемых артефактов}

Часто используемые артефакты должны кэшироваться для уменьшения времени работы с ними при использовании ресурсов из публичных репозиториев.

\subsubsection*{Для этого реализовать следующие требования:}

\begin{enumerate}
\item Кэширование артефактов для ускорения доступа к часто запрашиваемым элементам.
\item Управление шардами для распределения данных и оптимизации производительности.
\item Контроль доступа для обеспечения безопасности и соответствия политикам.
\item Функциональность поиска для быстрого нахождения артефактов внутри хранилища.
\item Высокая доступность для обеспечения непрерывности доступа к артефактам.
\item Масштабируемость для адаптации к изменяющимся объемам нагрузки.
\item Безопасность хранения и передачи артефактов для предотвращения несанкционированного доступа и атак.
\item Управление репозиториями для артефактов различных типов, таких как образы Docker, пакеты Maven, пакеты npm и другие.
\item Разрешение зависимостей для автоматического разрешения и загрузки зависимости, необходимые первичным артефактам.
\item Прокси-функциональность к другим внешним репозиториям артефактов, позволяя доступ к ним, как если бы они были локальными.
\item Неизменяемое кэширование для гарантии того, что однажды кэшированный артефакт не изменяется.
\item Проверка контрольных сумм для подтверждения целостности артефакта через проверку контрольных сумм для предотвращения использования поврежденных или подделанных двоичных файлов.
\item Оптимизация пропускной способности для уменьшения использования внешнего трафика за счет обслуживания артефактов из локального кэша по возможности.
\item Регулярные проверки консистентности для обеспечения синхронизации кэшированных артефактов с источниками восходящего потока.
\item Поддержка маркировки и простановки тегов артефактов для упрощения организации и извлечения.
\end{enumerate}


\subsection{Бот техподдержки}

У пользователей должна быть возможность получить быструю техподдрежку.

\subsubsection*{Для этого нужно:}

\begin{enumerate}
\item Разработать бота на базе ChatGPT для умной коммуникации с пользователем
\item У бота должна быть возможность взаимодействовать с API облака для диагностирования причин пользовательских проблем
\item Предоставить доступ ChatGPT к утилитам для диагностики: ping, ssh, dig
\item Предоставить доступ бота к документации и поиску по документации
\item Предоставить доступ бота к примерам Terraform и Ansible конфигурации
\item Реализовать возможность реконфигурации бота без редеплоя
\item Должна быть разработана база ответов на частые вопросы
\end{enumerate}

\subsection{K8s Service mesh (Istio)}

Взаимодействие по сети должно быть защищено для безопасности данных пользователей.

\subsubsection*{Для этого нужно реализовать следующее:}

\begin{enumerate}
\item Поддержка протокола HTTP/3 для обеспечения современного и эффективного веб-трафика.
\item Реализация взаимного TLS (mTLS) для обеспечения безопасности коммуникаций в gRPC, включая аутентификацию и шифрование.
\item Возможность перенаправления трафика для управления запросами и маршрутизации.
\item Внедрение мониторинга с использованием sidecar-контейнеров для сбора метрик и мониторинга ИС без изменения кода приложений.
\item Интеграция с системами SIEM и IDS через sidecar-контейнеры для обеспечения безопасности и мониторинга угроз в режиме реального времени.
\item Интеграция Fail2ban или аналогичного механизма для защиты от брутфорс-атак и других методов автоматизированных угроз.
\end{enumerate}

\subsection{Сервис хранения секретов пользователей\\ (Hashicorp Vault)}

У пользователя должна быть возможность хранить свои секреты в защищенном хранилище.

\subsubsection*{Нужно реализовать следующее:}

\begin{enumerate}
\item Использование алгоритма RAFT для обеспечения надежной и согласованной репликации данных.
\item Генерация динамических секретов по требованию с ограниченным временем жизни для доступа к системам, ИС или инструментам.
\item Автоматический или ручной отзыв секретов и учетных данных.
\item Надежное хранение секретов с шифрованием статичных данных и при передаче данных.
\item Поддержка версионирования ключей и значений с возможностью отката к предыдущим версиям.
\item Подробное ведение журналов действий, выполняемых в Vault, для аудита и контроля соответствия требованиям.
\item Предоставление услуг шифрования как ИС напрямую из Vault.
\item Механизмы аренды и продления срока действия секретов с автоматическим обновлением и отзывом.
\item Поддержка высокой доступности Vault с конфигурацией в нескольких центрах обработки данных.
\item Способность масштабирования для удовлетворения потребностей и обработки большого числа запросов.
\item Поддержка резервных узлов для обслуживания запросов только на чтение и замены основных узлов при их отказе.
\item Взаимодействие с Vault через HTTP API для интеграции с различными приложениями и ИС.
\item Расширение функционала Vault с помощью пользовательских плагинов или использование плагинов, разработанных сообществом.
\item Возможность расширения Vault новыми бэкендами секретов и методами аутентификации.
\item Мониторинг состояния и производительности Vault в реальном времени.
\item Интеграция с инструментами мониторинга, такими как Prometheus, Grafana, Splunk для получения аналитических данных и визуализации.
\item Защита экземпляра Vault с помощью механизма опечатывания, требующего кворум ключей для доступа к хранимым секретам.
\item Инструменты и процедуры для резервного копирования и восстановления Vault в случае катастрофы.
\end{enumerate}

\subsection{Автоматическая сборка облачных образов ОС}

Нужно реализовать автоматизацию процесса создания и обновления образов операционных систем для использования в облачной инфраструктуре, обеспечивая постоянную актуализацию безопасности и оптимизацию ресурсов.

\subsubsection*{Для этого должны быть реализованы следующие функции:}

\begin{enumerate}
\item Разработать процедуры тестирования для образов ОС, чтобы гарантировать их функциональность и стабильность.
\item Реализовать систему создания и удаления образов, позволяющую управлять жизненным циклом образа от разработки до отката.
\item Обеспечить настройку локальных зеркал в образах для ускорения процесса установки и обновления пакетов.
\item Включить функционал настройки таймзоны в образах, чтобы они соответствовали локальным настройкам пользователя.
\item Настроить процесс сборки для поддержки операционных систем Ubuntu, Debian и Rocky.
\item Интегрировать добавление агентов мониторинга и Gopnik в образы для наблюдения за состоянием и производительностью ОС на базе образов в облаке.
\end{enumerate}

\subsection{Ceph SDS}

Должна быть реализована интеграция с Ceph SDS для получения высокоэффективного, устойчивого к отказам и самовосстанавливающегося хранилища данных.

\subsubsection*{Для этого должны быть реализованы следующие требования:}

\begin{enumerate}
\item Поддержка CephFS для предоставления услуг файлового хранилища через сервис общего файлового хранилища OpenStack (Manila).
\item Реализация механизмов для обеспечения высокой доступности ИС.
\item Поддержка масштабируемости ИС для обработки увеличивающегося объема данных и запросов.
\item Внедрение Thin provisioning для эффективного распределения хранилища.
\item Реализация функции создания снапшотов и клонирования данных.
\item Поддержка автоматической репликации данных для повышения надежности хранения.
\item Реализация возможности автоматического или ручного перемещения данных между различными уровнями хранилища (SSD, HDD) в зависимости от моделей доступа.
\item Поддержка шифрования хранилища для неиспользуемых данных, что повышает безопасность конфиденциальных данных.
\item Введение управления квотами для контроля использования ресурсов хранения.
\item Средства для мониторинга производительности хранения.
\item Поддержка асинхронной репликации на вторичные сайты для целей восстановления после сбоев.
\end{enumerate}

\subsection{Система объектного хранения (Swift)}

Должна быть реализована адаптация OpenStack Swift, система хранения, оптимизированная для работы с неструктурированными данными.

\subsubsection*{Для этого должны быть реализованы следующие требования:}

\begin{enumerate}
\item Автоматическая репликация данных.
\item Возможность горизонтального масштабирования путём добавления новых узлов.
\item Обеспечение отсутствия единой точки отказа в архитектуре системы.
\item Интеграция с системой удостоверений OpenStack (Keystone) для аутентификации и контроля доступа.
\item Предоставление RESTful и S3 API для хранения и извлечения файлов с использованием стандартных HTTP-вызовов.
\item Поддержка высокой степени параллелизма обработки запросов.
\item Механизмы кэширования для ускорения чтения данных.
\item Возможность хранения и управления большими объектами путём их сегментации.
\item Возможность назначения пользовательских метаданных объектам и контейнерам.
\item Поддержка установки квот для контроля использования хранилища.
\item Интеграция Swift с различными компонентами OpenStack и предоставление доступа через SDK на разных языках программирования.
\item Поддержка использования ИС в качестве хранилища для целей резервного копирования и архивации.
\item Предоставление каждому проекту в Mekstack своей изолированной среды Swift для конфиденциальности и безопасности.
\item Интеграция с телеметрическими службами Mekstack для мониторинга и оповещения.
\item Генерация подробных журналов активности для интеграции с внешними инструментами управления логами.
\end{enumerate}

\subsection{Сервис предоставления хранилищ NFS (Manilla)}

Должна быть реализована адаптация NFSaaS (Manilla), чтобы пользователи облака могли создавать и управлять файловыми шарами (например, NFS или CIFS) без необходимости управлять нижележащими серверами файлового хранения или NAS (network-attached storage) устройствами

\subsubsection*{Для этого должны быть реализованы следующие требования:}

\begin{enumerate}
\item Поддержка NFS, SMB/CIFS и других популярных протоколов файлового обмена.
\item Возможность динамического создания и выделения файловых ресурсов по требованию пользователя без ручного вмешательства.
\item Обеспечение избыточной конфигурации (redundantly configured) хранения для поддержания доступности в случае сбоев оборудования или сети.
\item Автоматическое перемещение данных между различными уровнями хранения на основе моделей доступа для оптимизации стоимости и производительности.
\item Интеграция дедупликации данных для уменьшения объёма хранения путём удаления повторяющихся данных.
\item Сжатие данных в реальном времени для увеличения эффективной ёмкости хранения и снижения сетевой нагрузки.
\item Надёжное шифрование данных в покое и в процессе передачи для защиты конфиденциальной информации.
\item Возможность планирования и создания снапшотов и резервных копий по требованию для защиты данных и восстановления на определённый момент времени.
\item Подробные журналы всех действий для аудита соответствия и мониторинга безопасности.
\item API для интеграции с другими облачными ИС, приложениями и инструментами автоматизации.
\item Возможность устанавливать квоты на пользователя или группу для контроля потребления хранения.
\item Интеграция со службами каталогов, такими как LDAP или Active Directory для аутентификации пользователей.
\item Удобный веб-интерфейс для управления файловыми ресурсами, мониторинга использования и настройки параметров.
\item Периодические проверки здоровья и предсказывать анализ возможных отказов для обеспечения целостности системы.
\item Политики управления жизненным циклом данных, архивацией и хранением автоматически.
\item Детальные отчёты о потреблении хранения, активности пользователей и показателях производительности.
\item Настраиваемая система оповещений для различных событий, таких как превышение квот, проблемы со здоровьем системы или попытки несанкционированного доступа.
\end{enumerate}

\subsection{Сервис управления базами данных (DBaaS)}

Должна быть реализована адаптация DBaas (Trove), чтобы пользователи могли развертывать и управлять базами данных без необходимости заботиться о поддержке и управлении физическими серверами

\subsubsection*{Для этого должны быть реализованы следующие функции:}

\begin{enumerate}
\item Поддержка различных баз данных, таких как MySQL, PostgreSQL, MongoDB, Redis и Cassandra.
\item Настройка автоматических резервных копирований для защиты данных с возможностью планирования задач.
\item Возможность легкого восстановления баз данных из резервных копий пользователями.
\item Поддержка настройки репликации баз данных для повышения производительности и увеличения избыточности данных.
\item Предложение конфигураций для высокой доступности через механизмы репликации и кластеризации.
\item Возможность пользователей устанавливать и изменять параметры конфигурации для экземпляров баз данных.
\item Интеграция с телеметрическими службами Mekstack для предоставления метрик мониторинга производительности баз данных.
\item Предоставление предопределенных конфигураций баз данных для упрощения развертывания.
\item Возможность применения правил брандмауэра к экземплярам баз данных для обеспечения безопасности.
\item Возможность создания снимков баз данных для использования в будущем или для целей резервного копирования.
\item Предоставление RESTful API для всей функциональности, что позволяет автоматизацию и интеграцию с другими службами или приложениями.
\item Поддержка кластеризации для некоторых типов баз данных, позволяющая горизонтальное масштабирование для обработки больших нагрузок.
\item Использование OpenStack Cinder для хранения баз данных для обеспечения надежного и последовательного управления хранилищем.
\item Возможность изменения размера экземпляров баз данных на лету для корректировки ресурсов без значительного простоя.
\item Интеграция с ИС ведения журнала запросов для отслеживания запросов к базе данных в целях безопасности и оптимизации.
\item Поддержка различных версий хранилищ данных, позволяющая пользователям выбирать необходимые им версии.
\end{enumerate}

\subsection{Disaster Recovery сценарии}

Должны быть реализованы комплексные сценарии восстановления после сбоев (disaster recovery), гарантирующие минимальное время простоя и потерь данных в случае катастрофических событий или сбоев инфраструктуры

\subsubsection*{Для этого нужно сделать следующее:}

\begin{enumerate}
\item Разработка и проведение тестирования планов восстановления после сбоев в продуктовом окружении для проверки их надежности.
\end{enumerate}

\subsection{Managed K8s}

Нужно предоставить управляемый сервис Kubernetes, позволяющий пользователям легко развертывать, управлять и масштабировать контейнеризованные приложения с полной поддержкой и автоматизацией задач инфраструктуры

\subsubsection*{Для этого нужно сделать следующее:}

\begin{enumerate}
\item Предоставить пользователям возможность развертывания k8s кластеров минимальным количеством действий через пользовательский интерфейс.
\item Обеспечить возможность деплоя k8s кластеров через API для автоматизации процесса и интеграции с внешними системами.
\item Оптимизировать процесс установки k8s для работы с оборудованием, имеющим ограниченный объем оперативной памяти, для повышения эффективности использования ресурсов.
\item Внедрить систему мониторинга для отслеживания состояния кластеров k8s и ресурсов железа.
\item Настроить систему алертинга, которая будет уведомлять администраторов о проблемах и событиях в реальном времени.
\end{enumerate}

\subsection{Сервис отслеживания сверхутилизации ресурсов\\ (Generic Over-Provisioning Nuker and Inspection Kit)\\ (GOPNIK)}

Должен быть разработан сервис для определения и устранения неэффективно используемых ресурсов, например обнаружения выделенной виртуальной машине оперативной памяти, которую не используют.

\subsubsection*{Для этого должны быть реализованы следующие функции:}

\begin{enumerate}
    \item Отслеживание использования ресурсов
    \begin{enumerate}
        \item Мониторинг в реальном времени: Должен быть сделан мониторинг использования ЦП, памяти, хранения и ввода/вывода в реальном времени.
        \item Границы использования: Должны быть установлены границы для определения "недоиспользования" на основе процентных соотношений или конкретных метрик.
        \item Простановка тегов ресурсов: Должны быть проставлены теги для категоризации и мониторинга ресурсов по проекту, отделу или пользователю.
    \end{enumerate}
    \item Обнаружение аномалий
    \begin{enumerate}
        \item Анализ паттернов использования: Должны быть реализованы модели машинного обучения для изучения нормальных паттернов использования и обнаружения аномалий.
        \item Обнаружение неактивных ресурсов: Должны быть идентифицированы ресурсы с низкой или нулевой активностью за настраиваемый период времени.
    \end{enumerate}
    \item Мониторинг активности пользователей
    \begin{enumerate}
        \item Отслеживание входов: Должны быть отслежены входы пользователей в облачную среду и активные сессии.
        \item События развертывания и удаления: Должны быть отслежены события развертывания или удаления ресурсов пользователями.
    \end{enumerate}
    \item Управление затратами
    \begin{enumerate}
        \item Анализ затрат: Должны быть рассчитаны затраты на недоиспользованные ресурсы.
        \item Отчеты Showback и Chargeback: Должны быть созданы отчеты, показывающие распределение затрат между различными отделами или проектами при использовании облака.
    \end{enumerate}
    \item Система уведомлений
    \begin{enumerate}
        \item Оповещение: Должны быть отправлены уведомления пользователям или администраторам при недоиспользовании их ресурсов.
        \item Пользовательские правила уведомлений: Должна быть предоставлена возможность настройки условий, при которых срабатывают уведомления.
    \end{enumerate}
    \item Применение политик
    \begin{enumerate}
        \item Определение политик: Администраторы должны иметь возможность определять политики для обращения с недоиспользованными ресурсами.
        \item Автоматическое устранение: Должны быть реализованы автоматические действия, такие как остановка или освобождение ресурсов в соответствии с политиками.
    \end{enumerate}
    \item Отчетность и панели управления
    \begin{enumerate}
        \item Отчеты об использовании: Должны быть сгенерированы подробные отчеты об использовании ресурсов и случаях их занятия.
        \item Интерактивные панели управления: Должны быть предоставлены панели управления с фильтрами для различных параметров для визуализации использования ресурсов.
    \end{enumerate}
    \item Портал самообслуживания
    \begin{enumerate}
        \item Пользовательский интерфейс: Должен быть предложен удобный HTMX веб-интерфейс с поддержкой gRPC и HTTP/2 для пользователей для проверки их использования ресурсов.
        \item Самоустранение: Пользователи должны иметь возможность самостоятельно устранять проблемы с занятием ресурсов через портал.
    \end{enumerate}
    \item Масштабируемость и производительность
    \begin{enumerate}
        \item Масштабируемая архитектура: Сервис должен быть спроектирован таким образом, чтобы обрабатывать увеличивающееся количество ресурсов без снижения производительности.
        \item Минимальное влияние на производительность: Должно быть обеспечено, что процессы мониторинга и анализа оказывают минимальное воздействие на производительность облака.
    \end{enumerate}
\end{enumerate}

\subsection{Системы обнаружения и предотвращения вторжений\\ (IDS/IPS)}

Должны быть внедрены системы обнаружения и предотвращения вторжений для мониторинга и защиты облачной инфраструктуры от неавторизованного доступа, атак и уязвимостей.

\subsubsection*{Для этого нужно сделать следующее:}

\begin{enumerate}
\item Реализовать систему обнаружения аномалий сетевого трафика, способную идентифицировать необычные паттерны и потенциальные угрозы в реальном времени.
\item Интегрировать сигнатурный анализ трафика для обнаружения известных типов атак и вредоносных действий на основе базы данных сигнатур.
\item Предоставить возможности визуализации и составления отчетов для наглядного представления состояния сетевой безопасности и истории инцидентов.
\item Разработать процедуры реагирования на инциденты (incident response), включая автоматизированное принятие мер по предотвращению и минимизации ущерба от атак.
\item Внедрить систему создания алертов и уведомлений для оповещения администраторов и пользователей о безопасностных событиях и инцидентах в соответствии с настроенными правилами.
\end{enumerate}

\subsection{Система корреляции событий ИБ\\ (SIEM)}

Нужно развернуть систему SIEM для анализа событий безопасности в реальном времени и управления инцидентами, повышая общую безопасность и соответствие нормативным требованиям.

\subsubsection*{Для этого нужно сделать следующее:}

\begin{enumerate}
\item Обеспечить сбор событий информационных систем, включая логи операционных систем и журналы аутентификаций.
\item Реализовать интеграцию ИС SIEM с средствами защиты информации и приложениями для централизованного управления безопасностью.
\item Предусмотреть агрегацию и нормализацию данных из различных источников для унификации формата и упрощения анализа.
\item Внедрить корреляционные события для выявления связанных действий и потенциальных угроз, которые невозможно обнаружить при одиночном анализе.
\item Разработать функцию обогащения информации на внешних репутационных источниках для повышения точности и контекста событий безопасности.
\item Обеспечить надежное хранение данных событий на заданный период времени с возможностью гибкой настройки периода хранения.
\item Предоставить возможности интеграции через API для расширения функциональности и легкой интеграции с другими системами.
\item Интегрировать сервис SIEM с системами автоматической сборки (autobuilds) для отслеживания изменений и потенциальных уязвимостей в процессах разработки и развертывания.
\end{enumerate}

\subsection{Задачи менеджеров продукта}

Для успешной реализации продукта менеджеры должны выполнять следующие задачи:

\begin{enumerate}
\item Подготовка материалов для прохождения контроля проектным офисом.
\item Работа с пользователями (сбор обратной связи, CusDev).
\item Продвижение проекта, общение с потенциальными партнерами/инвесторами.
\item Интеграция с другими проектами МИЭМ.
\item Подготовка юзер гайдов.
\item Согласование финансирования.
\item Ведение переговоров в рамках вуза.
\item Контроль промежуточных точек проекта.
\item Проработка UX.
\end{enumerate}

\pagebreak

\section{Календарный план}

\newcolumntype{D}[1]{>{\centering\arraybackslash}p{#1}}

\begin{center}
\begin{longtable}{|D{0.4\textwidth}|D{0.2\textwidth}|D{0.3\textwidth}|}
\hline
Задание & Период работы & Исполнители \\
\hline
Monitoring & 31.10.2023 - 31.01.2024 & Services, SRE \\
\hline
Monitoring as a Service & 16.11.2023 - 29.04.2024 & Monitoring, SRE \\
\hline
NAT as a service & 16.10.2023 - 22.01.2024 & Services, SRE, Kubernetes, Frontend \\
\hline
Документация функциональности облака & 30.11.2023 - 29.04.2024 & ТехДок, SRE, Frontend \\
\hline
Caching Artifact Proxy & 16.10.2023 - 5.02.2024 & SRE \\
\hline
Сервис хранения секретов пользователей & 13.11.2023 - 25.12.2023 & SRE \\
\hline
Бот техподдержки & 11.10.2023 - 5.02.2024 & Services \\
\hline
GOPNIK & 20.11.2023 - 29.04.2024 & Services, SRE, Monitoring \\
\hline
Disaster Recovery сценарии & 8.02.2024 - 5.03.2024 & SRE \\
\hline
Автоматическая сборка облачных образов ОС & 11.12.2023 - 05.02.2024 & SRE \\
\hline
Ceph SDS & 20.11.2023 - 20.03.2024 & SRE \\
\hline
Object storage (Swift) & 01.01.2024 - 05.02.2024 & SRE \\
\hline
NFSaaS (Manilla) & 22.01.2024 - 23.02.2024 & SRE \\
\hline
DBaaS (Trove) & 25.01.2024 - 29.03.2024 & SRE \\
\hline
Managed K8s & 01.02.2024 - 01.04.2024 & SRE \\
\hline
IDS/IPS & 30.11.2023 - 29.04.2024 & InfoSec \\
\hline
SIEM & 30.11.2023 - 29.04.2024 & InfoSec \\
\hline
Подготовка материалов для прохождения контроля проектным офисом & 20.10.2023 - 8.11.2023 & Product Managers \\
\hline
Проработка UX & 01.12.2023 - 01.01.2024 & Product Managers \\
\hline
Написание юзер гайдов & 01.12.2023 - 01.01.2024 & Product Managers \\
\hline
Согласование финансирования & 01.01.2024 - 01.02.2024 & Product Managers \\
\hline
Работа с пользователями & 01.02.2024 - 01.03.2024 & Product Managers \\
\hline
Интеграция с другими проектами & 01.02.2024 - 01.03.2024 & Product Managers \\
\hline
Продвижение проекта, общение с потенциальными партнерами/инвесторами & 01.03.2024 - 01.04.2024 & Product Managers \\
\hline
\end{longtable}
\end{center}


\pagebreak

\section{Требования к отчетным материалам}

\subsubsection*{К разработанному решению должна быть написана сопутствующая документация}

\begin{enumerate}
\item Пользовательская документация
\item Документация разработчика
\item Отчет о проделанной работе
\item Документация из раздела “Документация функциональности облака”
\end{enumerate}

Документация должна быть выполнена на русском языке и соответствовать отраслевым стандартам.

\subsubsection*{Требования к пользовательской документации}

Документация пользователя должна содержать полную информацию, необходимую для использования продукта. В ней должны быть описаны страницы и внутренние разделы разработанного проекта, а также все функции, установленные в описании продукта. Представление данной информации не должно содержать неоднозначных толкований или ошибок. Каждый термин должен иметь один и тот же смысл во всех разделах документа.
Для верстки документации следует использовать LaTeX.

\subsubsection*{Требования к документации разработчика}

\begin{enumerate}
\item Документация должна быть читаемой и понятной для различных пользователей, таких как другие разработчики, тестировщики, менеджеры проекта и т.д.
\item Документация должна быть актуальной и содержать последнюю информацию о проекте, включая изменения и обновления.
\item Документация должна включать описание всех основных функций, а также детальную информацию о каждой функции, включая ее входные и выходные данные, алгоритмы, используемые при ее выполнении, любые ограничения и требования.
\item Документация должна быть организована логически и иметь хорошую структуру, чтобы пользователи могли быстро найти нужную информацию.
\item Документация должна быть размещена в доступном месте, чтобы другие разработчики и пользователи могли легко получить к ней доступ.
\end{enumerate}

\end{document}
