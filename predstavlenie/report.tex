\documentclass[14pt, a4paper]{extarticle}

\usepackage{polyglossia} % Поддержка многоязычности (fontspec подгружается автоматически)
\setmainlanguage[babelshorthands=true]{russian} % Язык по-умолчанию русский с поддержкой приятных команд пакета babel
\setotherlanguage{english} % Дополнительный язык = английский (в американской вариации по-умолчанию)
\setmonofont{CMU Typewriter Text} % моноширинный шрифт
\newfontfamily\cyrillicfonttt{CMU Typewriter Text} % моноширинный шрифт для кириллицы
\defaultfontfeatures{Ligatures=TeX} % стандартные лигатуры TeX, замены нескольких дефисов на тире и т. п. Настройки моноширинного шрифта должны идти до этой строки, чтобы при врезках кода программ в коде не применялись лигатуры и замены дефисов
\setmainfont{CMU Serif} % Шрифт с засечками
\newfontfamily\cyrillicfont{PT Astra Serif}[Script = Cyrillic] % Шрифт с засечками для кириллицы
\setsansfont{CMU Sans Serif} % Шрифт без засечек
\newfontfamily\cyrillicfontsf{CMU Sans Serif} % Шрифт без засечек для кириллицы

\usepackage{amsmath}
\usepackage{unicode-math}
\usepackage{float}
\usepackage[figurename=Рисунок]{caption, subcaption}
\usepackage{geometry}
\usepackage{graphicx}
\usepackage{titlesec}
\usepackage{indentfirst}
\usepackage[colorlinks=true,linkcolor=black,urlcolor=black,citecolor=black]{hyperref}
\usepackage{listings}
\usepackage{xcolor}
\usepackage[lighttt]{lmodern}
\usepackage{appendix}
\graphicspath{{./images/}}
\geometry{left=30mm,
          right=10mm,
          top=20mm,
          bottom=20mm}

\usepackage{fontspec}
\setmonofont[
  Contextuals={Alternate},
  Scale=MatchLowercase
]{Fira Mono for Powerline}
\lstdefinestyle{bash}{
    basicstyle=\ttfamily,
    framerule=10pt,
    breakatwhitespace=false,
    breaklines=true,
    captionpos=b,
    keepspaces=true,
    showspaces=false,
    showstringspaces=false,
    showtabs=false,
    tabsize=2,
    columns=fullflexible,
    literate={-}{-}1
}

\lstset{style=bash}
\setmathfont{Latin Modern Math}


\renewcommand{\figurename}{Рисунок}
\renewcommand{\lstlistingname}{Исходный код}

\DeclareSymbolFont{cyrletters}{\encodingdefault}{\familydefault}{m}{it}
\newcommand{\makecyrmathletter}[1]{%
\begingroup\lccode`a=#1\lowercase{\endgroup
\Umathcode`a}="0 \csname symcyrletters\endcsname\space #1
}
\count255="409
\loop\ifnum\count255<"44F
\advance\count255 by 1
\makecyrmathletter{\count255}
\repeat

\definecolor{mygray}{rgb}{0.3,0.3,0.3}

\let\oldtexttt\texttt
\renewcommand{\texttt}[1]{\textcolor{mygray}{\oldtexttt{#1}}}


\renewcommand{\thefigure}{\arabic{figure}}
\renewcommand{\theequation}{\arabic{figure}}
\renewcommand{\labelenumii}{\theenumii}
\renewcommand{\theenumii}{\theenumi.\arabic{enumii}.}

\date{\today}

\begin{document}

\begin{titlepage}
    \begin{center}
        Федеральное государственное автономное образовательное учреждение высшего
        образования\par
        НАЦИОНАЛЬНЫЙ ИССЛЕДОВАТЕЛЬСКИЙ УНИВЕРСИТЕТ\par ВЫСШАЯ ШКОЛА ЭКОНОМИКИ\par
        \vspace{1cm}
        Московский институт электроники и математики им. А.Н. Тихонова \\
        \vspace{1cm}
        Проект 19106 \\
        \vspace{2cm}
    \end{center}

    \begin{center}
        ТЕХНИЧЕСКОЕ ЗАДАНИЕ \\
        на разработку информационной системы \\
        \textbf{Mekstack: Приватное Облако} \\
    \end{center}
    \vfill

    \begin{minipage}[t]{0.5\textwidth}
        \begin{center}
        УТВЕРЖДАЮ
        \end{center}
        \begin{flushleft}
        Рыбаков Петр Владимирович\\
        Руководитель направления\\

        \vspace{0.5cm}
        Подпись
        \vspace{0.5cm}

        \today
        \end{flushleft}
    \end{minipage}%

    \vspace{2cm}

    \begin{minipage}[t]{0.5\textwidth}
        \begin{center}
        СОГЛАСОВАНО
        \end{center}
        \begin{flushleft}
        Емельяненко Максим Владимирович\\
        Руководитель проекта

        \vspace{0.5cm}
        Подпись
        \vspace{0.5cm}

        \today
        \end{flushleft}
    \end{minipage}%

    \vspace{2cm}
    \center
    Москва, 2023
\end{titlepage}

\pagenumbering{roman}

\tableofcontents

\pagebreak

\pagenumbering{arabic}

\section{Актуальность проекта}

В настоящее время большинство проектов требуют наличия специализированной виртуальной инфраструктуры, предоставляемой учебным заведением. Вопреки очевидным преимуществам, которые дает применение виртуальных вычислительных машин, это вызывает возникновение нового ряда требований со стороны пользователей, удовлетворение которых является целью данного проекта.

В рамках образовательного процесса студентов Московского института электроники и математики (МИЭМ) был проведен анализ, в результате которого были выявлены следующие требования к инфраструктуре и функционалу информационной системы (ИС), предоставляющей виртуальную инфраструктуру:

\begin{enumerate}

\item Разработка и внедрение интуитивно понятной и оперативной системы документации с функцией поиска, способной максимально раскрыть возможности частного облачного сервиса и обеспечить поддержку пользователей в освоении разнообразных методик облачных решений, включая наглядные примеры применения.
\item Создание высокодоступной системы, обладающей функционалом автоматического восстановления после сбоев (в том числе после отключения электроэнергии на территории всего кампуса), для обеспечения непрерывности работы облачных сервисов.
\item Интеграция современных технологических решений, используемых в передовых IT-компаниях, для предоставления студентам актуального практического опыта.
\item Реализация механизмов, позволяющих пользователям самостоятельно решать возникающие проблемы с минимальной зависимостью от службы технической поддержки и снижением нагрузки на персонал, ответственный за обработку запросов.
\item Обеспечение комплексного мониторинга состояния виртуальных машин и облачных сервисов с возможностью настройки пользовательских метрик и системы оповещений для своевременного реагирования на инциденты.
\item Разработка интерфейса или системы управления, которая дает возможность пользователям самостоятельно управлять ресурсами в облаке без вмешательства сторонних специалистов.
\item Внедрение возможности подключения к виртуальной инфраструктуре с использованием доменной авторизации Высшей школы экономики.
\item Предоставление функционала для самостоятельного создания клиентов OpenID с целью настройки доменной авторизации в собственных информационных системах.
\item Реализация возможности публикации разработанных информационных систем в сети Интернет.
\item Обеспечение доступа к оперативной технической поддержке для решения возникающих вопросов и отладки проблем.

\end{enumerate}

Преподавателям МИЭМ нужно предоставить

\begin{enumerate}
\item Возможность автоматизации сценариев для лабораторных работ, например, создание виртуальных машин.
\end{enumerate}

Сотрудникам МИЭМ требуется

\begin{enumerate}
\item Контроль используемых и заброшенных ресурсов.
\item Система логирования для отслеживания действий и изменений в их виртуальных средах, что обеспечит возможность аудита и поможет в диагностике проблем.
\item Система реагирования на инциденты.
\item Изолированная от критических узлов ВШЭ студенческая инфраструктура.
\item Снизить нагрузку на персонал, обслуживающий виртуальные машины.
\end{enumerate}


\end{document}
