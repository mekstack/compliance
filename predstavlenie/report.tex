\documentclass[14pt, a4paper]{extarticle}

\usepackage{polyglossia} % Поддержка многоязычности (fontspec подгружается автоматически)
\setmainlanguage[babelshorthands=true]{russian} % Язык по-умолчанию русский с поддержкой приятных команд пакета babel
\setotherlanguage{english} % Дополнительный язык = английский (в американской вариации по-умолчанию)
\setmonofont{CMU Typewriter Text} % моноширинный шрифт
\newfontfamily\cyrillicfonttt{CMU Typewriter Text} % моноширинный шрифт для кириллицы
\defaultfontfeatures{Ligatures=TeX} % стандартные лигатуры TeX, замены нескольких дефисов на тире и т. п. Настройки моноширинного шрифта должны идти до этой строки, чтобы при врезках кода программ в коде не применялись лигатуры и замены дефисов
\setmainfont{CMU Serif} % Шрифт с засечками
\newfontfamily\cyrillicfont{PT Astra Serif}[Script = Cyrillic] % Шрифт с засечками для кириллицы
\setsansfont{CMU Sans Serif} % Шрифт без засечек
\newfontfamily\cyrillicfontsf{CMU Sans Serif} % Шрифт без засечек для кириллицы

\usepackage{amsmath}
\usepackage{unicode-math}
\usepackage{float}
\usepackage[figurename=Рисунок]{caption, subcaption}
\usepackage{geometry}
\usepackage{graphicx}
\usepackage{titlesec}
\usepackage{indentfirst}
\usepackage[colorlinks=true,linkcolor=black,urlcolor=black,citecolor=black]{hyperref}
\usepackage{listings}
\usepackage{xcolor}
\usepackage[lighttt]{lmodern}
\usepackage{appendix}
\graphicspath{{./images/}}
\geometry{left=30mm,
          right=10mm,
          top=20mm,
          bottom=20mm}

\usepackage{fontspec}
\setmonofont[
  Contextuals={Alternate},
  Scale=MatchLowercase
]{Fira Mono for Powerline}
\lstdefinestyle{bash}{
    basicstyle=\ttfamily,
    framerule=10pt,
    breakatwhitespace=false,
    breaklines=true,
    captionpos=b,
    keepspaces=true,
    showspaces=false,
    showstringspaces=false,
    showtabs=false,
    tabsize=2,
    columns=fullflexible,
    literate={-}{-}1
}

\lstset{style=bash}
\setmathfont{Latin Modern Math}


\renewcommand{\figurename}{Рисунок}
\renewcommand{\lstlistingname}{Исходный код}

\DeclareSymbolFont{cyrletters}{\encodingdefault}{\familydefault}{m}{it}
\newcommand{\makecyrmathletter}[1]{%
\begingroup\lccode`a=#1\lowercase{\endgroup
\Umathcode`a}="0 \csname symcyrletters\endcsname\space #1
}
\count255="409
\loop\ifnum\count255<"44F
\advance\count255 by 1
\makecyrmathletter{\count255}
\repeat

\definecolor{mygray}{rgb}{0.3,0.3,0.3}

\let\oldtexttt\texttt
\renewcommand{\texttt}[1]{\textcolor{mygray}{\oldtexttt{#1}}}


\renewcommand{\thefigure}{\arabic{figure}}
\renewcommand{\theequation}{\arabic{figure}}
\renewcommand{\labelenumii}{\theenumii}
\renewcommand{\theenumii}{\theenumi.\arabic{enumii}.}

\date{\today}

\begin{document}

\begin{titlepage}
    \begin{center}
        Федеральное государственное автономное образовательное учреждение высшего
        образования\par
        НАЦИОНАЛЬНЫЙ ИССЛЕДОВАТЕЛЬСКИЙ УНИВЕРСИТЕТ\par ВЫСШАЯ ШКОЛА ЭКОНОМИКИ\par
        \vspace{1cm}
        Московский институт электроники и математики им. А.Н. Тихонова \\
        \vspace{1cm}
        Проект 19106 \\
        \vspace{2cm}
    \end{center}

    \begin{center}
        ТЕХНИЧЕСКОЕ ЗАДАНИЕ \\
        на разработку информационной системы \\
        \textbf{Mekstack: Приватное Облако} \\
    \end{center}
    \vfill

    \begin{minipage}[t]{0.5\textwidth}
        \begin{center}
        УТВЕРЖДАЮ
        \end{center}
        \begin{flushleft}
        Рыбаков Петр Владимирович\\
        Руководитель направления\\

        \vspace{0.5cm}
        Подпись
        \vspace{0.5cm}

        \today
        \end{flushleft}
    \end{minipage}%

    \vspace{2cm}

    \begin{minipage}[t]{0.5\textwidth}
        \begin{center}
        СОГЛАСОВАНО
        \end{center}
        \begin{flushleft}
        Емельяненко Максим Владимирович\\
        Руководитель проекта

        \vspace{0.5cm}
        Подпись
        \vspace{0.5cm}

        \today
        \end{flushleft}
    \end{minipage}%

    \vspace{2cm}
    \center
    Москва, 2023
\end{titlepage}

\pagenumbering{roman}

\tableofcontents

\pagebreak

\pagenumbering{arabic}

\section{Актуальность проекта}

Сегодня редко какой проект обходится без необходимости в выделенной виртуальной
инфраструктуры, которая предоставляется университетом. Несмотря на очевидную
пользу, которую приносит использование виртуальной инфраструктуры, это также
порождает новый пласт потребностей пользователей, который и призван решать
данный проект.

Студентам МИЭМ нужно:

\begin{enumerate}

\item Наличие удобной и быстрой документации с функцией поиска, которая позволит в полной мере реализовать потенциал приватного облака и поможет пользователям ориентироваться в различных подходах облачных решений с примерами.
\item Высокодоступная система, которая в том числе умеет автоматически восстанавливаться после внезапного отключения (disaster recovery), например после отключения электричества во всем кампусе.
\item Поддержка современных технологий (Database as a Service, Managed Kubernetes, S3, Terraform), которые используются в современных IT-компаниях и позволят студентам получить необходимый опыт работы.
\item Более самостоятельный подход к решению проблем, чтобы минимизировать зависимость от техподдержки и снизить нагрузку на сотрудников, отвечающих за обработку заявок.
\item Комплексный мониторинг состояния виртуальных машин и облачных сервисов, с возможностью кастомизации метрик и настройки системы уведомлений для своевременного реагирования на проблемы.
\item Интерфейс или система управления, которая позволила бы пользователям самостоятельно управлять своими ресурсами в облаке без посторонней помощи.
\item Возможность подключаться к виртуальной инфраструктуре с доменной авторизацией ВШЭ.
\item Возможность самостоятельно создавать OpenID клиенты для настройки доменной авторизации на своих информационных система.
\item Возможность публиковать свои информационные системы в интернете.
\item Доступ к техподдержке для решения возникающих проблем.

\end{enumerate}

Преподавателям МИЭМ нужно

\begin{enumerate}
\item Возможность автоматизации сценариев для лабораторных работ, например, создание виртуальных машин.
\end{enumerate}

\begin{enumerate}
\item Сотрудникам МИЭМ нужно.
\item Контроль используемых и заброшенных ресурсов.
\item Система логирования для отслеживания действий и изменений в их виртуальных средах, что обеспечит возможность аудита и поможет в диагностике проблем.
\item Система реагирования на инциденты.
\item Изолированная от критических узлов ВШЭ студенческая инфраструктура.
\item Снизить нагрузку на персонал, обслуживающий виртуальные машины.
\end{enumerate}


\end{document}
