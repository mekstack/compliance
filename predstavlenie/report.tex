\documentclass[14pt, a4paper]{extarticle}

\usepackage{polyglossia} % Поддержка многоязычности (fontspec подгружается автоматически)
\setmainlanguage[babelshorthands=true]{russian} % Язык по-умолчанию русский с поддержкой приятных команд пакета babel
\setotherlanguage{english} % Дополнительный язык = английский (в американской вариации по-умолчанию)
\setmonofont{CMU Typewriter Text} % моноширинный шрифт
\newfontfamily\cyrillicfonttt{CMU Typewriter Text} % моноширинный шрифт для кириллицы
\defaultfontfeatures{Ligatures=TeX} % стандартные лигатуры TeX, замены нескольких дефисов на тире и т. п. Настройки моноширинного шрифта должны идти до этой строки, чтобы при врезках кода программ в коде не применялись лигатуры и замены дефисов
\setmainfont{CMU Serif} % Шрифт с засечками
\newfontfamily\cyrillicfont{PT Astra Serif}[Script = Cyrillic] % Шрифт с засечками для кириллицы
\setsansfont{CMU Sans Serif} % Шрифт без засечек
\newfontfamily\cyrillicfontsf{CMU Sans Serif} % Шрифт без засечек для кириллицы

\usepackage{amsmath}
\usepackage{unicode-math}
\usepackage{float}
\usepackage[figurename=Рисунок]{caption, subcaption}
\usepackage{geometry}
\usepackage{graphicx}
\usepackage{titlesec}
\usepackage{indentfirst}
\usepackage[colorlinks=true,linkcolor=black,urlcolor=black,citecolor=black]{hyperref}
\usepackage{listings}
\usepackage{xcolor}
\usepackage[lighttt]{lmodern}
\usepackage{appendix}
\graphicspath{{./images/}}
\geometry{left=30mm,
          right=10mm,
          top=20mm,
          bottom=20mm}

\usepackage{fontspec}
\setmonofont[
  Contextuals={Alternate},
  Scale=MatchLowercase
]{Fira Mono for Powerline}
\lstdefinestyle{bash}{
    basicstyle=\ttfamily,
    framerule=10pt,
    breakatwhitespace=false,
    breaklines=true,
    captionpos=b,
    keepspaces=true,
    showspaces=false,
    showstringspaces=false,
    showtabs=false,
    tabsize=2,
    columns=fullflexible,
    literate={-}{-}1
}

\lstset{style=bash}
\setmathfont{Latin Modern Math}


\renewcommand{\figurename}{Рисунок}
\renewcommand{\lstlistingname}{Исходный код}

\DeclareSymbolFont{cyrletters}{\encodingdefault}{\familydefault}{m}{it}
\newcommand{\makecyrmathletter}[1]{%
\begingroup\lccode`a=#1\lowercase{\endgroup
\Umathcode`a}="0 \csname symcyrletters\endcsname\space #1
}
\count255="409
\loop\ifnum\count255<"44F
\advance\count255 by 1
\makecyrmathletter{\count255}
\repeat

\definecolor{mygray}{rgb}{0.3,0.3,0.3}

\let\oldtexttt\texttt
\renewcommand{\texttt}[1]{\textcolor{mygray}{\oldtexttt{#1}}}


\renewcommand{\thefigure}{\arabic{figure}}
\renewcommand{\theequation}{\arabic{figure}}
\renewcommand{\labelenumii}{\theenumii}
\renewcommand{\theenumii}{\theenumi.\arabic{enumii}.}

\date{\today}

\begin{document}

\begin{titlepage}
    \begin{center}
        Федеральное государственное автономное образовательное учреждение высшего
        образования\par
        НАЦИОНАЛЬНЫЙ ИССЛЕДОВАТЕЛЬСКИЙ УНИВЕРСИТЕТ\par ВЫСШАЯ ШКОЛА ЭКОНОМИКИ\par
        \vspace{1cm}
        Московский институт электроники и математики им. А.Н. Тихонова \\
        \vspace{1cm}
        Проект 19106 \\
        \vspace{2cm}
    \end{center}

    \begin{center}
        ТЕХНИЧЕСКОЕ ЗАДАНИЕ \\
        на разработку информационной системы \\
        \textbf{Mekstack: Приватное Облако} \\
    \end{center}
    \vfill

    \begin{minipage}[t]{0.5\textwidth}
        \begin{center}
        УТВЕРЖДАЮ
        \end{center}
        \begin{flushleft}
        Рыбаков Петр Владимирович\\
        Руководитель направления\\

        \vspace{0.5cm}
        Подпись
        \vspace{0.5cm}

        \today
        \end{flushleft}
    \end{minipage}%

    \vspace{2cm}

    \begin{minipage}[t]{0.5\textwidth}
        \begin{center}
        СОГЛАСОВАНО
        \end{center}
        \begin{flushleft}
        Емельяненко Максим Владимирович\\
        Руководитель проекта

        \vspace{0.5cm}
        Подпись
        \vspace{0.5cm}

        \today
        \end{flushleft}
    \end{minipage}%

    \vspace{2cm}
    \center
    Москва, 2023
\end{titlepage}

\pagenumbering{roman}

\tableofcontents

\pagebreak

\pagenumbering{arabic}

\section{Актуальность проекта}

В настоящее время большинство проектов требуют наличия специализированной виртуальной инфраструктуры, предоставляемой учебным заведением. Вопреки очевидным преимуществам, которые дает применение виртуальных вычислительных машин, это вызывает возникновение нового ряда требований со стороны пользователей, удовлетворение которых является целью данного проекта.

В рамках образовательного процесса студентов Московского института электроники и математики (МИЭМ) был проведен анализ, в результате которого были выявлены следующие требования к инфраструктуре и функционалу информационной системы (ИС), предоставляющей виртуальную инфраструктуру:

\begin{enumerate}

\item Разработка и внедрение интуитивно понятной и оперативной системы документации с функцией поиска, способной максимально раскрыть возможности частного облачного сервиса и обеспечить поддержку пользователей в освоении разнообразных методик облачных решений, включая наглядные примеры применения.
\item Создание высокодоступной системы, обладающей функционалом автоматического восстановления после сбоев (в том числе после отключения электроэнергии на территории всего кампуса), для обеспечения непрерывности работы облачных сервисов.
\item Интеграция современных технологических решений, используемых в передовых IT-компаниях, для предоставления студентам актуального практического опыта.
\item Реализация механизмов, позволяющих пользователям самостоятельно решать возникающие проблемы с минимальной зависимостью от службы технической поддержки и снижением нагрузки на персонал, ответственный за обработку запросов.
\item Обеспечение комплексного мониторинга состояния виртуальных машин и облачных сервисов с возможностью настройки пользовательских метрик и системы оповещений для своевременного реагирования на инциденты.
\item Разработка интерфейса или системы управления, которая дает возможность пользователям самостоятельно управлять ресурсами в облаке без вмешательства сторонних специалистов.
\item Внедрение возможности подключения к виртуальной инфраструктуре с использованием доменной авторизации Высшей школы экономики.
\item Предоставление функционала для самостоятельного создания клиентов OpenID с целью настройки доменной авторизации в собственных информационных системах.
\item Реализация возможности публикации разработанных информационных систем в сети Интернет.
\item Обеспечение доступа к оперативной технической поддержке для решения возникающих вопросов и отладки проблем.

\end{enumerate}

Преподавателям МИЭМ нужно

\begin{enumerate}
\item Обеспечить функцию автоматизации стандартных операций для проведения лабораторных работ, включая, но не ограничиваясь, автоматизированное создание виртуальных машин
\end{enumerate}

Сотрудникам МИЭМ требуется

\begin{enumerate}
\item Реализация функции контроля за использованием и бездействующими ресурсами информационных технологий.
\item Создание системы регистрации событий, которая позволит фиксировать все действия и изменения в виртуальной среде, что станет основой для проведения аудита и будет способствовать эффективной диагностике проблем.
\item Внедрение системы оперативного реагирования на инциденты информационной безопасности.
\item Организовать студенческую инфраструктуру, функционирующую в изоляции от критически важных элементов инфраструктуры Высшей школы экономики.
\item Способ по снижению нагрузки на персонал, отвечающий за обслуживание виртуальных машин.
\end{enumerate}

\section{Назначение результатов проекта}

Назначение результатов проекта заключается в решении следующих проблем:

\begin{enumerate}
\item Автоматизация процесса обработки заявок для сокращения времени их рассмотрения и исключения необходимости ручного ввода данных.
\item Внедрение системы самообслуживания для пользователей, позволяющей им самостоятельно управлять выделенными ресурсами, что сократит нагрузку на персонал техподдержки.
\item Создание комплексной системы мониторинга, обеспечивающей постоянный контроль за состоянием виртуальных машин, облачной инфраструктуры и информационных систем как в целом, так и пользовательских.
\item Реализация функции логирования всех значимых событий для обеспечения возможности аудита и упрощения диагностики проблем.
\item Упрощение эксплуатации облачной инфраструктуры за счет разработки инструкций по работе с виртуальными машинами и облачными сервисами.
\item Улучшение пользовательского интерфейса с целью сделать его более интуитивно понятным и удобным для пользователей.
\item Интеграция с системами Московского института электроники и математики и личным кабинетом Высшей школы экономики для создания единого информационного пространства.
\item Оптимизация использования ресурсов путем идентификации и обработки заброшенных виртуальных машин и дисков.
\end{enumerate}

\section{Цель разработки}

\begin{enumerate}
\item Создание автоматизированного, легко конфигурируемого и расширяемого кластера серверов на основе проекта OpenStack Kayobe.
\item Создание системы централизованного управления и мониторинга ресурсов.
\item Интеграция с личным кабинетом ВШЭ.
\item Подготовка подробной документации (документация администратора, архитектурная документация в формате Docs as a Code) и пользовательских мануалов, настройка Elasticsearch для поиска по документации.
\item Создание VPN сервиса для подключения студентов к сети облака.
\item Создание ИС для выделения доменов и создания tls-сертификатов пользователям.
\item Интеграция с Ceph SDS для получения высокоэффективного, устойчивого к отказам и самовосстанавливающегося хранилища данных.
\item Адаптация и внедрение OpenStack Swift, системой хранения, оптимизированной для работы с неструктурированными данными.
\item Адаптация и внедрение NFSaaS (Manilla), чтобы пользователи облака могли создавать и управлять файловыми шарами (например, NFS или CIFS) без необходимости управлять нижележащими серверами файлового хранения или NAS (network-attached storage) устройствами.
\item Адаптация и внедрение DBaas (Trove), чтобы пользователи могли развертывать и управлять базами данных без необходимости заботиться о поддержке и управлении физическими серверами.
\item Разработать прокси-сервис для кэширования и предоставления общего доступа к артефактов, который обеспечит оптимизацию доступа к зависимостям и библиотекам для ускорения процессов сборки и развертывания.
\item Разработать бота на базе ChatGPT для умной коммуникации с пользователем.
\item Реализовать сервисную сетку (service mesh) на базе Istio внутри кластера Kubernetes для обеспечения управления трафиком, безопасности, и наблюдаемости ИС с целью повышения надёжности и управляемости микросервисной архитектуры.
\item Предоставить централизованный, защищённый сервис для хранения, управления и контроля доступа к секретам и конфиденциальной информации, используя HashiCorp Vault как сервис в облачной инфраструктуре.
\item Реализовать облачный сервис удостоверяющего центра (IdP) для централизованного управления идентификацией и аутентификацией пользователей с использованием интеграции с системой Vault.
\item Автоматизировать процесс создания и обновления образов операционных систем для использования в облачной инфраструктуре, обеспечивая постоянную актуализацию безопасности и оптимизацию ресурсов.
\item Разработать и реализовать комплексные сценарии восстановления после сбоев (disaster recovery), гарантирующие минимальное время простоя и потерь данных в случае катастрофических событий или сбоев инфраструктуры.
\item Внедрить системы обнаружения и предотвращения вторжений для мониторинга и защиты облачной инфраструктуры от неавторизованного доступа, атак и уязвимостей.
\item Развернуть систему SIEM для анализа событий безопасности в реальном времени и управления инцидентами, повышая общую безопасность и соответствие нормативным требованиям.
\item Предоставить управляемый сервис Kubernetes, позволяющий пользователям легко развертывать, управлять и масштабировать контейнеризованные приложения с полной поддержкой и автоматизацией задач инфраструктуры.
\item Разработать и интегрировать модуль удостоверяющего центра для Keystone, который позволит упростить и централизовать доменную аутентификацию и авторизацию пользователей в Mekstack.
\end{enumerate}

\section{Требования к результатам проекта}

\subsection{Интеграция авторизации с личным кабинетом MIEM}

Нужно разработать и интегрировать модуль удостоверяющего центра для Keystone, который позволит упростить и централизовать аутентификацию и авторизацию пользователей в Mekstack.

\paragraph{Триггер начала работ} Успешная имплементация Apache Kafka в личном кабинете МИЭМ.
\paragraph{Требования} Сервис должен создавать и изменять пользователей и проекты в OpenStack аналогично их сущностям в кабинете ВШЭ.
\paragraph{Задачи}

\begin{enumerate}
\item Разработать пользовательский модуль аутентификации Keystone, который будет использовать данные проектов пользователя из утверждений JWT OpenID вместо больших маппингов для аутентификации.
\item Написать Создать патч для Keystone.
\item Написать пайплайн сборки и публикации образов.
\item Написать пайплайн для периодического слияния нашей версии Keystone с основной кодовой базой Keystone.
\item Написать юнит-тесты для патча.
\item Настроить логирование для нового модуля аутентификации.
\item Реализовать канареечное развертывание для поэтапного внедрения модуля.
\item Разработать пайплайн для сборки образов Keystone с новым модулем.
\item Автоматизировать публикацию образов в реестр.
\item Обеспечить версионирование для возможности отката и ведения аудита.
\item Интегрировать пайплайн с существующими процессами CI/CD.
\item Провести аудит безопасности нового модуля аутентификации.
\item Устранить уязвимости, обнаруженные в ходе аудита.
\item Гарантировать соответствие соответствующим стандартам безопасности и лучшим практикам.
\item Разработать стратегию внедрения нового модуля в существующие среды Keystone.
\item Создать инструмент миграции для пользователей, использующих предыдущий метод аутентификации.
\item Протестировать процесс миграции в контролируемых условиях.
\end{enumerate}

\end{document}
